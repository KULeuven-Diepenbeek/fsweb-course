% Options for packages loaded elsewhere
\PassOptionsToPackage{unicode}{hyperref}
\PassOptionsToPackage{hyphens}{url}
%
\documentclass[
]{article}
\usepackage{amsmath,amssymb}
\usepackage{iftex}
\ifPDFTeX
  \usepackage[T1]{fontenc}
  \usepackage[utf8]{inputenc}
  \usepackage{textcomp} % provide euro and other symbols
\else % if luatex or xetex
  \usepackage{unicode-math} % this also loads fontspec
  \defaultfontfeatures{Scale=MatchLowercase}
  \defaultfontfeatures[\rmfamily]{Ligatures=TeX,Scale=1}
\fi
\usepackage{lmodern}
\ifPDFTeX\else
  % xetex/luatex font selection
\fi
% Use upquote if available, for straight quotes in verbatim environments
\IfFileExists{upquote.sty}{\usepackage{upquote}}{}
\IfFileExists{microtype.sty}{% use microtype if available
  \usepackage[]{microtype}
  \UseMicrotypeSet[protrusion]{basicmath} % disable protrusion for tt fonts
}{}
\makeatletter
\@ifundefined{KOMAClassName}{% if non-KOMA class
  \IfFileExists{parskip.sty}{%
    \usepackage{parskip}
  }{% else
    \setlength{\parindent}{0pt}
    \setlength{\parskip}{6pt plus 2pt minus 1pt}}
}{% if KOMA class
  \KOMAoptions{parskip=half}}
\makeatother
\usepackage{xcolor}
\usepackage{color}
\usepackage{fancyvrb}
\newcommand{\VerbBar}{|}
\newcommand{\VERB}{\Verb[commandchars=\\\{\}]}
\DefineVerbatimEnvironment{Highlighting}{Verbatim}{commandchars=\\\{\}}
% Add ',fontsize=\small' for more characters per line
\newenvironment{Shaded}{}{}
\newcommand{\AlertTok}[1]{\textcolor[rgb]{1.00,0.00,0.00}{\textbf{#1}}}
\newcommand{\AnnotationTok}[1]{\textcolor[rgb]{0.38,0.63,0.69}{\textbf{\textit{#1}}}}
\newcommand{\AttributeTok}[1]{\textcolor[rgb]{0.49,0.56,0.16}{#1}}
\newcommand{\BaseNTok}[1]{\textcolor[rgb]{0.25,0.63,0.44}{#1}}
\newcommand{\BuiltInTok}[1]{\textcolor[rgb]{0.00,0.50,0.00}{#1}}
\newcommand{\CharTok}[1]{\textcolor[rgb]{0.25,0.44,0.63}{#1}}
\newcommand{\CommentTok}[1]{\textcolor[rgb]{0.38,0.63,0.69}{\textit{#1}}}
\newcommand{\CommentVarTok}[1]{\textcolor[rgb]{0.38,0.63,0.69}{\textbf{\textit{#1}}}}
\newcommand{\ConstantTok}[1]{\textcolor[rgb]{0.53,0.00,0.00}{#1}}
\newcommand{\ControlFlowTok}[1]{\textcolor[rgb]{0.00,0.44,0.13}{\textbf{#1}}}
\newcommand{\DataTypeTok}[1]{\textcolor[rgb]{0.56,0.13,0.00}{#1}}
\newcommand{\DecValTok}[1]{\textcolor[rgb]{0.25,0.63,0.44}{#1}}
\newcommand{\DocumentationTok}[1]{\textcolor[rgb]{0.73,0.13,0.13}{\textit{#1}}}
\newcommand{\ErrorTok}[1]{\textcolor[rgb]{1.00,0.00,0.00}{\textbf{#1}}}
\newcommand{\ExtensionTok}[1]{#1}
\newcommand{\FloatTok}[1]{\textcolor[rgb]{0.25,0.63,0.44}{#1}}
\newcommand{\FunctionTok}[1]{\textcolor[rgb]{0.02,0.16,0.49}{#1}}
\newcommand{\ImportTok}[1]{\textcolor[rgb]{0.00,0.50,0.00}{\textbf{#1}}}
\newcommand{\InformationTok}[1]{\textcolor[rgb]{0.38,0.63,0.69}{\textbf{\textit{#1}}}}
\newcommand{\KeywordTok}[1]{\textcolor[rgb]{0.00,0.44,0.13}{\textbf{#1}}}
\newcommand{\NormalTok}[1]{#1}
\newcommand{\OperatorTok}[1]{\textcolor[rgb]{0.40,0.40,0.40}{#1}}
\newcommand{\OtherTok}[1]{\textcolor[rgb]{0.00,0.44,0.13}{#1}}
\newcommand{\PreprocessorTok}[1]{\textcolor[rgb]{0.74,0.48,0.00}{#1}}
\newcommand{\RegionMarkerTok}[1]{#1}
\newcommand{\SpecialCharTok}[1]{\textcolor[rgb]{0.25,0.44,0.63}{#1}}
\newcommand{\SpecialStringTok}[1]{\textcolor[rgb]{0.73,0.40,0.53}{#1}}
\newcommand{\StringTok}[1]{\textcolor[rgb]{0.25,0.44,0.63}{#1}}
\newcommand{\VariableTok}[1]{\textcolor[rgb]{0.10,0.09,0.49}{#1}}
\newcommand{\VerbatimStringTok}[1]{\textcolor[rgb]{0.25,0.44,0.63}{#1}}
\newcommand{\WarningTok}[1]{\textcolor[rgb]{0.38,0.63,0.69}{\textbf{\textit{#1}}}}
\setlength{\emergencystretch}{3em} % prevent overfull lines
\providecommand{\tightlist}{%
  \setlength{\itemsep}{0pt}\setlength{\parskip}{0pt}}
\setcounter{secnumdepth}{-\maxdimen} % remove section numbering
\ifLuaTeX
  \usepackage{selnolig}  % disable illegal ligatures
\fi
\usepackage{bookmark}
\IfFileExists{xurl.sty}{\usepackage{xurl}}{} % add URL line breaks if available
\urlstyle{same}
\hypersetup{
  pdftitle={2. JDBC en JDBI},
  hidelinks,
  pdfcreator={LaTeX via pandoc}}

\title{2. JDBC en JDBI}
\author{}
\date{}

\begin{document}
\maketitle

\subsection{1.1 Java Database Connectivity
(JDBC)}\label{java-database-connectivity-jdbc}

\subsubsection{1.1.1 Hoe verbind ik Java met de
DB?}\label{hoe-verbind-ik-java-met-de-db}

\href{https://www.tutorialspoint.com/jdbc/index.htm}{JDBC} is een
interface in de JDK die ons in staat stelt om een connectie te openen
naar een database. \textbf{JDBC is een API}: een abstracitelaag of een
\emph{protocol}. Dit betekent dat we met JDBC kunnen verbinden naar
eender welke server van eender welke flavor: een Oracle SQL, MSSQL, of
SQLite database. De database \emph{vendor} wordt verborgen achter de
JDBC laag. Voor deze oefeningen beperken we ons tot SQLite.

\{\{\% notice info \%\}\} Update van de cursustekst en oefeningen zijn
onderweg, tegen de volgende les van 29/03 zullen we alles overschakelen
naar MYSQL dat jullie eerder al gebruikten met XAMP en phpmyadmin. Enige
verschillen met MYSQL zijn: \textbar{} MYSQL \textbar{} SQLITE
\textbar{} \textbar---\textbar---\textbar{} \textbar{} AUTO\_INCREMENT
\textbar{} AUTOINCREMENT \textbar{} \textbar{} BOOLEAN \textbar{} BOOL
\textbar{} \{\{\% /notice \%\}\}

Voor elke database moet er dus een vendor-specifieke driver als
dependency worden toegevoegd. In het geval van SQLite is dit de
\href{https://github.com/xerial/sqlite-jdbc}{sqlite-jdbc} driver, de
\texttt{sqlite-jdbc} package. JDBC zelf leeft in \texttt{java.sql} en is
een integraal onderdeel van de JDK: dit moeten we dus niet apart
oplijsten als dependency of downloaden.

\{\{\}\} graph LR; Java{[}Java{]} JDBC{[}JDBC{]} SQLITE{[}SQLite-JDBC{]}
DB{[}(SQLite Database){]} subgraph Java space subgraph JDK Java
-.-\textgreater{} JDBC end JDBC --\textgreater{} SQLITE end subgraph DB
space SQLITE --\textgreater{} DB end \{\{\textless{} /mermaid
\textgreater\}\}

De \texttt{sqlite-jdbc} package zorgt voor de brug tussen onze Java
applicatie en de database, maar we spreken die aan via JDBC.

Enkele belangrijke statements:

\begin{enumerate}
\def\labelenumi{\arabic{enumi}.}
\tightlist
\item
  Een connectie naar een database vastleggen:
  \texttt{var\ connection\ =\ DriverManager.getConnection("jdbc:sqlite:mydb.db");}
\item
  Een \texttt{SELECT} query uitvoeren:
  \texttt{var\ s\ =\ connection.createStatement();\ var\ result\ =\ s.executeQuery("...");\ var\ cell\ =\ result.getString("column");}
\item
  Een \texttt{INSERT}/\texttt{UPDATE}/\ldots{} query uitvoeren (die de
  structuur of inhoud van de database \textbf{wijzigt}):
  \texttt{var\ s\ =\ connection.createStatement();\ s.executeUpdate("...");}
\end{enumerate}

Het volgende voorbeeld opent een verbinding naar een DB, maakt een tabel
aan, voegt een record toe, en telt het aantal records:

\begin{Shaded}
\begin{Highlighting}[]
\KeywordTok{private} \BuiltInTok{Connection}\NormalTok{ connection}\OperatorTok{;}
\KeywordTok{public} \DataTypeTok{void} \FunctionTok{createDb}\OperatorTok{()} \KeywordTok{throws} \BuiltInTok{SQLException} \OperatorTok{\{}
\NormalTok{    connection }\OperatorTok{=} \BuiltInTok{DriverManager}\OperatorTok{.}\FunctionTok{getConnection}\OperatorTok{(}\StringTok{"jdbc:sqlite:mydb.db"}\OperatorTok{);}
    \DataTypeTok{var}\NormalTok{ s }\OperatorTok{=}\NormalTok{ connection}\OperatorTok{.}\FunctionTok{createStatement}\OperatorTok{();}
\NormalTok{    s}\OperatorTok{.}\FunctionTok{executeUpdate}\OperatorTok{(}\StringTok{"CREATE TABLE mijntabel(nr INT); INSERT INTO mijntabel(nr) VALUES(1);"}\OperatorTok{)}
\NormalTok{    s}\OperatorTok{.}\FunctionTok{close}\OperatorTok{();}
\OperatorTok{\}}
\KeywordTok{public} \DataTypeTok{void} \FunctionTok{verifyDbContents}\OperatorTok{()} \KeywordTok{throws} \BuiltInTok{SQLException} \OperatorTok{\{}
    \DataTypeTok{var}\NormalTok{ s }\OperatorTok{=}\NormalTok{ connection}\OperatorTok{.}\FunctionTok{createStatement}\OperatorTok{();}
    \DataTypeTok{var}\NormalTok{ result }\OperatorTok{=}\NormalTok{ s}\OperatorTok{.}\FunctionTok{executeQuery}\OperatorTok{(}\StringTok{"SELECT COUNT(*) FROM mijntabel;"}\OperatorTok{);}
    \DataTypeTok{var}\NormalTok{ count }\OperatorTok{=}\NormalTok{ result}\OperatorTok{.}\FunctionTok{getInt}\OperatorTok{(}\DecValTok{0}\OperatorTok{);}
\NormalTok{    s}\OperatorTok{.}\FunctionTok{close}\OperatorTok{();}

    \ControlFlowTok{assert}\NormalTok{ count }\OperatorTok{==} \DecValTok{1}\OperatorTok{;}
\OperatorTok{\}}
\end{Highlighting}
\end{Shaded}

\textbf{Gradle} dependency: laatste versie van
\href{https://mvnrepository.com/artifact/org.xerial/sqlite-jdbc}{sqlite-jdbc
in mvnrepository.com}.

Merk op dat \texttt{SQLException} een \textbf{checked exception} is die
je constant moet meespelen in de method signature of expliciet moet
opvangen. Het probleem van een
\texttt{try\ \{\ \}\ catch\ \{\ \}\ finally\ \{\ \}} block is dat in de
finally je ook geen \texttt{close()} kan uitvoeren zonder opnieuw een
\texttt{try} block te openen\ldots{} Inception!

Het \texttt{connection.close()} statement moet er voor zorgen dat voor
elke request de connection netjes wordt afgesloten. Een database heeft
meestal een \textbf{connection pool} van x aantel beschikbare
connections, bijvoorbeeld 5. Als een connection per request niet wordt
gesloten, heeft de voglende bezoeker van onze website geen enkele kans
om zijn zoekquery te lanceren, omdat de database dan zegt dat alle
connecties zijn opgebruikt!

Merk op dat de String \texttt{jdbc:sqlite:mydb.db} een \textbf{lokale
SQLite database file} aanmaakt op het huidig relatief pad, zodat je met
SQLite Explorer data kan inspecteren. Deze file wordt herbruikt: indien
je een tabel aanmaakt de eerste keer, gaat dit de tweede keer crashen
met \emph{table already exists}. Houd hier dus rekening mee (e.v.t. met
\texttt{IF\ NOT\ EXISTS}). Je kan ook een \textbf{in-memory database}
aanmaken, die volledig in RAM leeft en bij elke opstart opnieuw wordt
aangemaakt, met de String \texttt{jdbc:sqlite:memory}.

Werk je met een andere database maar heb je geen idee hoe die speciale
connection string te vormen? Geen probleem, daarvoor dient
\url{https://www.connectionstrings.com/}. Bijvoorbeeld, een connectie
naar de Microsoft Azure cloud kan met de volgende syntax:

\begin{verbatim}
Server=tcp:myserver.database.windows.net,1433;Database=myDataBase;User ID=mylogin@myserver;Password=myPassword;Trusted_Connection=False;Encrypt=True;
\end{verbatim}

Het is de connection string die bepaalt welke dependency binding
gebruikt wordt! Dit noemen we \emph{late binding}: er is \textbf{geen
expliciete referentie} naar iets van SQLite in de Java code; we werken
\emph{enkel} met JDBC zelf. Als je de vendor driver vergeet toe te
voegen als Gradle dependency gebeurt er dit:

\begin{verbatim}
Exception in thread "main" java.sql.SQLException: No suitable driver found for jdbc:sqlite:mydb.db
    at java.sql/java.sql.DriverManager.getConnection(DriverManager.java:702)
    at java.sql/java.sql.DriverManager.getConnection(DriverManager.java:251)
    at Demo.main(Demo.java:8)
\end{verbatim}

\{\{\% notice note \%\}\} In-memory databases (ConStr.
\texttt{jdbc:sqlite:memory}), die met een lege database vertrekken, en
constant \texttt{CREATE\ TABLE()} statements issuen, vervuilen je
broncode. Als je veel SQL moet uitvoeren is het beter om dit in een
\texttt{.sql} bestand te bewaren in \texttt{src/main/resources} en
eenmalig in te lezen als SQL met
\texttt{new\ String(Files.readAllBytes(Paths.g));}, om te kunnen
uitvoeren via \texttt{statement.executeUpdate()}. Zie het
\href{https://github.com/KULeuven-Diepenbeek/db-course/tree/main/examples/java/jdbc-repo-start}{jdbc-repo-start}
project op GitHub als voorbeeld. \{\{\% /notice \%\}\}

\subsubsection{1.1.2 Queries/Objecten in
JDBC}\label{queriesobjecten-in-jdbc}

Stel dat we dezelfde studenten willen inladen in een \texttt{Student}
klasse instantie: van de \texttt{TABLE\ STUDENT} naar de
\texttt{class\ Student}. In geval van JDBC is dat veel handwerk:

\begin{enumerate}
\def\labelenumi{\arabic{enumi}.}
\tightlist
\item
  Maak een verbinding met de database.
\item
  Voer de \texttt{SELECT} statements uit.
\item
  Loop door de \texttt{ResultSet} en maak een nieuwe \texttt{Student}
  instantie aan. Vang alle mogelijke fouten zelf op: wat met lege
  kolommen, \texttt{null}? Wat met \texttt{INTEGER} kolommen die je wilt
  mappen op een \texttt{String} property?
\end{enumerate}

Om van de huidige resultatenrij naar de volgende te springen in
\texttt{ResultSet} gebruikt men de methode \texttt{next()} in een
typisch \texttt{while()} formaat:

\begin{Shaded}
\begin{Highlighting}[]
\DataTypeTok{var}\NormalTok{ result }\OperatorTok{=}\NormalTok{ statement}\OperatorTok{.}\FunctionTok{executeQuery}\OperatorTok{(}\StringTok{"SELECT * FROM iets"}\OperatorTok{);}
\ControlFlowTok{while}\OperatorTok{(}\NormalTok{result}\OperatorTok{.}\FunctionTok{next}\OperatorTok{())} \OperatorTok{\{}
    \DataTypeTok{var}\NormalTok{ eenString }\OperatorTok{=}\NormalTok{ result}\OperatorTok{.}\FunctionTok{getString}\OperatorTok{(}\StringTok{"kolomnaam"}\OperatorTok{);}
    \CommentTok{// doe iets!}
\OperatorTok{\}}
\end{Highlighting}
\end{Shaded}

Zie ook
\href{https://docs.oracle.com/javase/7/docs/api/java/sql/ResultSet.html}{ResultSet
Oracle Javadoc}.

Aangezien we reeds hebben kennis gemaakt met de (beperkte) API,
schakelen we onmiddellijk over naar de oefeningen:

\subsubsection{1.1.3 Oefeningen}\label{oefeningen}

Gebruik voor de oefeningen de \texttt{student} tabel statements uit
\href{/transacties/failures-rollbacks/}{RDBMS Transacties - Failures \&
Rollbacks}.

\begin{enumerate}
\def\labelenumi{\arabic{enumi}.}
\tightlist
\item
  Maak (én test!) een klasse \texttt{StudentRepository} die de volgende
  methode implementeert. Zoals je ziet is het de bedoeling dat de JDBC
  \texttt{Connection} instance elders wordt aangemaakt, bijvoorbeeld in
  een \textbf{aparte} \texttt{ConnectionManager} klasse.
\end{enumerate}

\begin{Shaded}
\begin{Highlighting}[]
\KeywordTok{public} \KeywordTok{class}\NormalTok{ StudentRepository }\OperatorTok{\{}
        \KeywordTok{public} \FunctionTok{StudentRepository}\OperatorTok{(}\BuiltInTok{Connection}\NormalTok{ connection}\OperatorTok{);}
        \KeywordTok{public} \BuiltInTok{List}\OperatorTok{\textless{}}\NormalTok{Student}\OperatorTok{\textgreater{}} \FunctionTok{getStudentsByName}\OperatorTok{(}\BuiltInTok{String}\NormalTok{ name}\OperatorTok{);}
\OperatorTok{\}}
\end{Highlighting}
\end{Shaded}

\begin{enumerate}
\def\labelenumi{\arabic{enumi}.}
\setcounter{enumi}{1}
\tightlist
\item
  Hoe zou je bovenstaande \texttt{StudentRepository} unit (integratie)
  testen, zonder de ``productie database'' op te vullen met testdata?
  (Hint: kijk naar het constructor argument). Hoe kan je
  \texttt{getStudentsByName()} testen zonder de volgende oefening
  afgewerkt te hebben, die nieuwe studenten bewaren pas mogelijk maakt?
\item
  Breid dit uit met \texttt{saveNewStudent(Student)}.
\item
  Breid dit uit met \texttt{updateStudent(Student)}. Wat moet je doen
  als deze student nog niet in de database zit? Welke gegevens update je
  wel en welke niet?
\item
  Merk op dat elke keer als je je project opstart je geen
  \texttt{CREATE\ TABLE\ student} kan uitvoeren als je een file-based
  SQLite bestand hanteert: eens de tabel is aangemaakt geeft een nieuwe
  create foutmeldingen. \texttt{DROP\ TABLE\ IF\ EXISTS\ student;} lost
  dit op, maar daardoor ben je ook altijd je data kwijt. Hoe los je dit
  probleem op?
\item
  Stel dat een \texttt{Student} is ingeschreven in een \texttt{Cursus}
  met properties \texttt{naam} (vb. ``databases'') en \texttt{ects} (vb.
  4).

  \begin{itemize}
  \tightlist
  \item
    Maak een \texttt{CursusRepository} om nieuwe cursussen te bewaren.
  \item
    Hoe link je de \texttt{Student} klasse met de \texttt{Cursus}
    klasse? wat verandert er in de query van
    \texttt{getStudentsByName()}?
  \end{itemize}
\end{enumerate}

\textbf{Tips}:

\begin{itemize}
\tightlist
\item
  \texttt{executeUpdate()} van een \texttt{Statement} is erg omslachtig
  als je een string moet stamenstellen die een \texttt{INSERT} query
  voorstelt (haakjes, enkele quotes, \ldots). Wat meer is, als de input
  van een UI komt, kan dit gehacked worden, door zelf de quote te
  sluiten in de string. Dit noemt men \textbf{SQL Injection}, en om dat
  te vermijden gebruik je in JDBC de \texttt{prepareStatement()}
  methode. Zie
  \href{https://docs.oracle.com/javase/tutorial/jdbc/basics/prepared.html}{JDBC
  Basics: Prepared Statements}. De String die je meegeeft bevat in de
  plaats van parameters een vraagteken:
  \texttt{INSERT\ INTO\ STUDENT(bla,\ bla)\ VALUES(?,\ ?)}. Die
  parameters vul je daarna aan met
  \texttt{preparedStatement.setString()} of \texttt{setInt()}. Op die
  manier is de code zowel \emph{netjes} als \emph{injectie-vrij}!
\item
  Als je data wenst op te halen dat is verspreid over verschillende
  tabellen, is de kans groot dat een \texttt{JOIN} SQL statement nodig
  is. Probeer eerst de query te schrijven in de \emph{SQLite DB Browser}
  tool. De Java objecten opvullen is de laatste taak.
\end{itemize}

\end{document}
